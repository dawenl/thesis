%!TEX root = thesis.tex

Bayesian probabilistic modeling provides a powerful framework for building flexible models to incorporate latent structures through likelihood model and prior. When we specify a model, we make certain assumptions about the underlying data-generating process with respect to these latent structures. For example, the latent Dirichlet allocation (LDA) model assumes that when generating a document, we first select a latent \textit{topic} and then select a word that often appears in the selected \textit{topic}. We can uncover the latent structures conditioned on the observed data via posterior inference. In this thesis, we apply the tools of probabilistic latent variable models and try to understand complex real-world data about music semantics and user behavior. 

We first look into the problem of automatic music tagging -- inferring the semantic tags (e.g., ``jazz'', ``piano'', ``happy'', etc.) from the audio features. We treat music tagging as a matrix completion problem and apply the Poisson matrix factorization model jointly on the vector-quantized audio features and a ``bag-of-tags'' representation. This approach exploits the shared latent structure between semantic tags and acoustic codewords. We present experimental results on the Million Song Dataset for both annotation and retrieval tasks, illustrating the steady improvement in performance as more data is used. 

We then move to the intersection between music semantics and user behavior: music recommendation. The leading performance in music recommendation is achieved by collaborative filtering methods which exploit the similarity patterns in user's listening history. We address the fundamental \textit{cold-start} problem of collaborative filtering: it cannot recommend new songs that no one has listened to. We train a neural network on semantic tagging information as a content model and use it as a prior in a collaborative filtering model. The proposed system is evaluated on the Million Song Dataset and shows comparably better result than the collaborative filtering approaches, in addition to the favorable performance in the \textit{cold-start} case. 

Finally, we focus on general recommender systems. We examine two different types of data: implicit and explicit feedback, and introduce the notion of \textit{user exposure} (whether or not a user is exposed to an item) as part of the data-generating process, which is latent for implicit data and observed for explicit data. For implicit data, we propose a probabilistic matrix factorization model and infer the user exposure from data. In the language of causal analysis \citep{imbens2015causal}, \textit{user exposure} has close connection to the assignment mechanism. We leverage this connection more directly for explicit data and develop a causal inference approach to recommender systems. We demonstrate that causal inference for recommender systems leads to improved generalization to new data.

Exact posterior inference is generally intractable for latent variables models. Throughout this thesis, we will design specific inference procedure to tractably analyze the large-scale data encountered under each scenario. 

