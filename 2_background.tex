
%!TEX root = thesis.tex

\chapter{Background}\label{chpt:background}

In this chapter, we discuss some background knowledge and previous work helpful to understanding the rest of the dissertation. 
Broadly speaking, we will make use of inference techniques for probabilistic modeling, collaborative filtering method for recommender systems, and causal inference. We give a high-level overview of the three fields and introduce some necessary definitions that will be used in the subsequent chapters. 

\section{Probabilistic modeling and inference techniques}\label{chpt:background:sec:inference}

We begin by defining the general problem setup for probabilistic latent variable models. We observe data $\mbx = \{x_1, \dots, x_N\}$. We assume the data is generated stochastically by a model $p(\mbx \g \mbz, \theta)$ that is governed by some latent variables $\mbz = \{z_1, \dots, z_N\}$ as well as model parameters $\mb\theta$\footnote{The distinction between latent variables $\mbz$ and model parameters $\mb\theta$ can be somewhat arbitrary. Here we follow the convention that the dimensionality of the latent variables grows with the number of observations (hence both $\mbx$ and $\mbz$ are indexed by $n$), while that of model parameters does not. Latent variables and model parameters loosely correspond to \textit{local variables} and \textit{global variables} in the notation of \citet{hoffman2013stochastic}}. We can also incorporate priors $p(\mbz, \mb\theta)$. We leave the dependency structure of prior generic. 

\begin{figure}[ht]
  \centering
     \begin{tikzpicture}

  % Define nodes
  \node[obs] 			(x) {$x_n$};
  \node[latent, above=.8cm of x] (z) {$z_n$};
  \node[latent, left=.6cm of z] (theta) {$\mb\theta$};

  % Connect the nodes
  \edge {z} {x} ; %
  \edge {theta} {x};
  \edge[densely dotted] {theta}{z};

  % Plates
  \plate {xz} {(x)(z)} {$N$};

\end{tikzpicture}


  \caption{Graphical model representation for the general latent variable models.}
\label{chpt:background:fig:general}
\end{figure}

\Cref{chpt:background:fig:general} demonstrates the graphical model representation for the general latent variable models described above. Shaded nodes represent observed variables. Unshaded nodes represent
hidden (unobserved) variables. A directed edge from node $a$ to node $b$
denotes that the variable $b$ depends on the value of variable
$a$. Plates denote replication by the value in the lower corner
of the plate. We use doted line to indicate that the dependency between model parameters $\mb\theta$ and latent variables $\mbz$ are optional -- it depends on how the prior $p(\mbz, \mb\theta)$ is specified. 

In Bayesian inference, the
goal is to reason about the posterior distribution over the model parameters and latent variables conditioned on the data, which is given by Bayes' rule:
\begin{equation}
p(\mb{\theta}, \mbz \g \mbx) = \frac{p(\mbx \g \mbz, \mb{\theta}) p(\mbz, \mb{\theta})}{ p(\mbx)} = \frac{p(\mbx \g \mbz, \mb{\theta}) p(\mbz, \mb{\theta})}{ \int_\mb{\theta} \int_\mbz p(\mbx \g \mbz, \mb{\theta}) p(\mbz, \mb{\theta}) \dif \mbz \dif \mb\theta}
\label{chpt:background:eq:bayes}
\end{equation}
Through posterior inference, we are able to uncover the latent structure induced by the model. Except in very simple models, posterior $p(\mb{\theta}, \mbz \g \mbx)$ is generally intractable to compute due to the normalizing constant $p(\mbx)$ which requires computing the integral in the denominator of \Cref{chpt:background:eq:bayes}. In practice, people normally resort to \gls{MCMC} methods \citep{neal1993probabilistic,robert2013monte} to obtain samples from the posterior distribution to form a Monte Carlo estimator about the predictive quantities. 

\PP General inference procedure: maximum likelihood estimation or maximum a posteriori. 

\PP MLE: 
\begin{equation*}
\mb\theta^\textrm{MLE} = \argmax_\mb\theta \log p(\mbx \g \mb\theta) = \argmax_\mb\theta \log \int_\mbz p(\mbx, \mbz \g \mb\theta) \dif\mbz
\end{equation*}

\PP MAP:
\begin{equation*}
\mb\theta^\textrm{MAP} = \argmax_\mb\theta \log p(\mb\theta, \mbz \g \mbx) = \argmax_\mb\theta \log \int_\mbz p(\mb\theta, \mbz, \mbx) \dif\mbz
\end{equation*}

\PP Examples: Gaussian mixture models, hidden Markov models, LDA.

\subsection{Parameter estimation via expectation-maximization}\label{chpt:background:sec:em}

\PP \gls{EM} \citep{dempster1977maximum}

One of the problems with directly maximizing the observable data likelihood, as shown above, is that the summation is inside the logarithm. So what if we move on to the complete data likelihood $\mathrm{P}(\mathbf{X, Z})$ and then marginalize the hidden variable $\mathbf{Z}$? Let's do the derivation\footnote{There are actually a lot of different versions of derivations for EM algorithm. The one presented here gives the most intuition to the author.}.

Start from the log-likelihood:
\[
\log\mathrm{P}(\mathbf{X}|\boldsymbol{\theta}) = \log\int_{\mathbf{Z}} \mathrm{P}(\mathbf{X, Z}|\boldsymbol{\theta}) \dif{\mbZ}
\]
Here we make use of the variational point of view by adding a variational distribution $q(\mathbf{Z})$ and the fact that logarithm function is concave:
\begin{align*}
\log\mathrm{P}(\mathbf{X}|\boldsymbol{\theta}) =& \log\Big(\sum_{\mathbf{Z}} \frac{\mathrm{P}(\mathbf{X, Z}|\boldsymbol{\theta})}{q(\mathbf{Z})} q(\mathbf{Z})\Big) \\
\geq& \sum_{\mathbf{Z}}q(\mathbf{Z}) \log\Big(\frac{\mathrm{P}(\mathbf{X, Z}|\boldsymbol{\theta})}{q(\mathbf{Z})}\Big)
\end{align*}
By making use of the Jensen's inequality, we move out the summation from the logarithm successfully. We could of course find out $q(\mathbf{Z})$ by exploring when the equality holds for Jensen's inequality. However, we will solve it from a different approach here. We first want to learn how much we have lost from the Jensen's inequality:
\begin{align*}
\Delta =& \log\mathrm{P}(\mathbf{X}|\boldsymbol{\theta}) - \sum_{\mathbf{Z}}q(\mathbf{Z}) \log\Big(\frac{\mathrm{P}(\mathbf{X, Z}|\boldsymbol{\theta})}{q(\mathbf{Z})}\Big) \\
=& \sum_{\mathbf{Z}}q(\mathbf{Z})\log\mathrm{P}(\mathbf{X}|\boldsymbol{\theta}) - \sum_{\mathbf{Z}}q(\mathbf{Z}) \log\Big(\frac{\mathrm{P}(\mathbf{X, Z}|\boldsymbol{\theta})}{q(\mathbf{Z})}\Big)\\
=& \sum_{\mathbf{Z}}q(\mathbf{Z})\log\Big(\frac{\mathrm{P}(\mathbf{X}|\boldsymbol{\theta}) q(\mathbf{Z}) }{\mathrm{P}(\mathbf{X, Z}|\boldsymbol{\theta})} \Big)\\
=& \sum_{\mathbf{Z}}q(\mathbf{Z})\log\Big(\frac{q(\mathbf{Z})}{\mathrm{P}(\mathbf{Z}|\mathbf{X}, \boldsymbol{\theta})} \Big)\\
=& \text{KL}(q\|p)
\end{align*}
Thus, the difference is actually the KL-divergence between the variational distribution $q(\mathbf{Z})$ and the posterior distribution $\mathrm{P}(\mathbf{Z}|\mathbf{X}, \boldsymbol{\theta})$. Thus, we could rearrange the log-likelihood as:
\[
\log\mathrm{P}(\mathbf{X} | \boldsymbol{\theta}) = \underbrace{\sum_{\mathbf{Z}}q(\mathbf{Z}) \log\Big(\frac{\mathrm{P}(\mathbf{X, Z}|\boldsymbol{\theta})}{q(\mathbf{Z})}\Big)}_{\mathcal{L}(q, \boldsymbol{\theta})} + \text{KL}(q\|p)
\]
Since the KL-divergence is non-negative for any $q$ and $p$, $\mathcal{L}(q, \boldsymbol{\theta})$ acts as a lower-bound for the log-likelihood. Let's denote the log-likelihood as $\ell(\boldsymbol{\theta)}$ for simplicity. 

EM algorithm has 2 steps as its name suggests: Expectation(E) step and Maximization(M) step.

\textbf{In the E step}, from the variational point of view, our goal is to choose a proper distribution $q(\mathbf{Z})$ such that it best approximates the log-likelihood. At this moment, we have existing parameter $\boldsymbol{\theta}^{\text{old}}$.
Thus we set the variational distribution $q(\mathbf{Z})$ equal the posterior distribution $\mathrm{P}(\mathbf{Z}|\mathbf{X}, \boldsymbol{\theta}^{\text{old}})$ so that $\text{KL}(q\|p) = 0$. In that case, we make the lower-bound $\mathcal{L}(q, {\boldsymbol{\theta}})$ equal $\ell(\boldsymbol{\theta})$.

\textbf{In the M step}, we have $q(\mathbf{Z})$ fixed and maximize the $\mathcal{L}(q, \boldsymbol{\theta})$, which is equivalent to maximize $\ell(\boldsymbol{\theta})$,  w.r.t. the parameter $\boldsymbol{\theta}$. Unless we reach the convergence, the lower-bound $\mathcal{L}(q, \boldsymbol{\theta})$ will increase with the new parameter $\boldsymbol{\theta}^{\text{new}}$. Since the parameter $\boldsymbol{\theta}$ changes from the E step, KL-divergence no longer equals 0, which creates gap between $\mathcal{L}(q, \boldsymbol{\theta})$ and $\ell(\boldsymbol{\theta})$ again. And this gap will be filled out in the next E step.

To see analytically the objective function in M step, substitute $q(\mathbf{Z})$ with $\mathrm{P}(\mathbf{Z}|\mathbf{X}, \boldsymbol{\theta}^{\text{old}})$ from E step:
\begin{align*}
\mathcal{L}(q, \boldsymbol{\theta}) =& \sum_{\mathbf{Z}}\mathrm{P}(\mathbf{Z}|\mathbf{X}, \boldsymbol{\theta}^{\text{old}}) \log\Big(\frac{\mathrm{P}(\mathbf{X, Z}|\boldsymbol{\theta})}{\mathrm{P}(\mathbf{Z}|\mathbf{X}, \boldsymbol{\theta}^{\text{old}})}\Big) \\
=& \underbrace{\sum_{\mathbf{Z}}\mathrm{P}(\mathbf{Z}|\mathbf{X}, \boldsymbol{\theta}^{\text{old}}) \log\mathrm{P}(\mathbf{X, Z}|\boldsymbol{\theta})}_{\mathcal{Q}(\boldsymbol{\theta}, \boldsymbol{\theta}^{\text{old}})} - \mathcal{H}\{\mathrm{P}(\mathbf{Z|X},\boldsymbol{\theta}^{\text{old}})\}
\end{align*} 
where $\mathcal{H}\{\mathrm{P}(\mathbf{Z|X},\boldsymbol{\theta}^{\text{old}})\}$ represents the negative entropy of $\mathrm{P}(\mathbf{Z|X},\boldsymbol{\theta}^{\text{old}})$, which is irrelevant to the parameter $\boldsymbol{\theta}$. Thus we could consider it as a constant. What really matters is $\mathcal{Q}(\boldsymbol{\theta}, \boldsymbol{\theta}^{\text{old}})$ which we could view as the expectation of $\mathrm{P}(\mathbf{X, Z}|\boldsymbol{\theta})$ under the posterior distribution $\mathrm{P}(\mathbf{Z}|\mathbf{X}, \boldsymbol{\theta}^{\text{old}})$. There are a few very nicely-drawn figures, visualizing the whole procedures of EM algorithm in Chapter 9 of \cite{Bishop:2006:PRM:1162264}. 

The sweet spot in M step is that, instead of directly maximizing $\mathrm{P}(\mathbf{X})$ which does not have a close form solution, computing $\mathrm{P}(\mathbf{X, Z})$ is generally much simpler because we can just view it as a model with no hidden variables.

EM algorithm is usually referred as a typical example of \emph{coordinate ascent}, where in each E/M step, we have one variable fixed ($\boldsymbol{\theta}^{\text{old}}$ in E step and $q(\mathbf{Z})$ in M step), and maximize w.r.t. another one. Coordinate ascent is widely used in numerical optimization.


\subsection{Variational inference}\label{chpt:background:sec:vi}

\PP Variational inference is a deterministic alternative to \gls{MCMC} methods \citep{jordan1999introduction,wainwright2008graphical,blei2016variational}. The basic idea behind variational inference is to choose a tractable family of variational distributions $q(\mbz)$ to approximate the intractable posterior $p(\mbz, \theta | \mbx)$, so that the Kullback-Leibler
(KL) divergence between the variational distribution and the true posterior $KL(q \| p)$ is minimized.

\PP Mean-field variational inference chooses from the completely factorized mean-field family: $q(\mbz) = \prod_d q(z_d)$.

\PP Variational inference is closed related to \gls{EM} algorithm. 

\section{Recommender systems}\label{chpt:background:sec:recsys}

Making good recommendations is an important problem on the web. In the
recommendation problem, we observe how a set of users interacts with a
set of items, and our goal is to show each user a set of previously
unseen items that she will like.  Broadly speaking, recommender
systems use historical data to infer users' preferences, and then use
the inferred preferences to suggest items.  Good recommender
systems are essential as the web grows; users are overwhelmed with
choice.

\subsection{Explicit and implicit feedback} \label{chpt:background:sec:data}

Traditionally there are two modes of this problem, recommendation from
explicit data and recommendation from implicit data.  With explicit
data, users rate some items (positively, negatively, or along a
spectrum) and we can predict their missing ratings (the task of \textit{rating prediction}, popularized by the Netflix Prize\footnote{\url{http://www.netflixprize.com/}}). This is called explicit data because it is enough to only use the rated items to infer a user's
preferences. Positively rated items indicate types of items that she
likes; negatively rated items indicate items that she does not like. Explicit data is of great value, but it is often difficult to obtain. 

In implicit data, each user expresses a binary decision about items\footnote{In principle, implicit data can go beyond binary: For example, the number of times a user listened to certain songs can also be considered as implicit feedback. However, in practice we find that the binary indicator of \textit{interaction} tends to carry the most signal. }---for
example this can be clicking, purchasing, viewing---and we aim to
predict unclicked items that she would want to click on. Unlike
ratings data, implicit data is easily accessible.  While ratings data
requires action on the part of the users, implicit data is often a
natural byproduct of their behavior, e.g., browsing histories, click
logs, and past purchases. We will explore recommender systems for both implicit and explicit data in this dissertation in \Cref{chpt:expomf} and \Cref{chpt:causal_rec}, respectively.

\subsection{Collaborative filtering for recommender systems} \label{chpt:background:sec:cf}

\PP Collaborative filtering is the workhorse of recommender systems.

\PP Neighborhood-based \citep{sarwar2001item} and latent factor model \citep{koren2009matrix}.

\parhead{Matrix factorization for collaborative filtering.} User-item preference data, whether explicit or implicit, can be encoded in a user by item matrix. Throughout this dissertation, a user is indexed by $u \in \{1, \dots, U\}$, an item is indexed by $i \in \{1, \dots, I\}$, and we will refer to this user by item matrix as the \emph{click matrix} or the \emph{interaction matrix}. Given the observed entries in this matrix $\{y_{ui}: (u, i) \in \cO\}$, the recommendation task is often framed as filling in the unobserved entries.  Matrix factorization models, which infer (latent) user preferences and item attributes by factorizing the click matrix, are standard in recommender systems \citep{koren2009matrix}. 

\PP Add a cartoon about the latent space from matrix factorization

From a generative modeling perspective they
can be understood as first drawing user and item latent factors corresponding,
respectively, to user preferences and item attributes. Then drawing 
observations from a specific distribution (e.g., a Poisson
or a Gaussian) with its mean parametrized by the dot product between the user and
the item factors. Formally, Gaussian matrix factorization is \citep{mnih2007probabilistic}: 
\begin{equation} 
\begin{split}
	\boldsymbol\theta_{u} &\sim \mathcal{N}(\bzero, \lambda_\theta^{-1} \mathrm{I}_K) \quad \textrm{for } u = 1, \dots, U, \\
	\boldsymbol\beta_{i} &\sim \mathcal{N}(\bzero, \lambda_\beta^{-1} \mathrm{I}_K) \quad \textrm{for } i = 1, \dots, I, \\
	y_{ui} &\sim \mathcal{N}(\boldsymbol\theta_u^\top \boldsymbol\beta_i, \lambda_y^{-1}) \quad \textrm{for } (u, i) \in \cO, 
 \end{split}
 \label{chpt:background:eq:gmf}
 \end{equation}
where $\boldsymbol\theta_u$ and $\boldsymbol\beta_i$ represent user $u$'s latent preferences and
item $i$'s attributes respectively. We use the mean and covariance to
parametrize the Gaussian distribution. $\lambda_\theta$, $\lambda_\beta$, and
$\lambda_y$ can be treated as hyperparameters, or be given priors for a full Bayesian treatment. $\mathrm{I}_K$ stands for the identity
matrix of dimension $K$. The maximum \emph{a posteriori} estimates of the Gaussian matrix factorization model is equivalent to the solution of minimizing the squared loss between the estimated and actual preferences $\sum_{(u, i)\in \mathcal{O}} (y_{ui} - \theta_u^\top\beta_i)^2$ with $\ell_2$ regularization on the latent factors.

\parhead{Collaborative filtering for implicit data.} \Cref{chpt:background:eq:gmf} can be equally applied to both explicit and implicit data. The real difference is how to define the observed set $\cO$: For explicit data, it can simply be the user-item pairs where user $u$ has clicked on (rated) item $i$. However, we can not copy the same definition for implicit data. The reason is that the data is binary and thus, when inferring a user's preferences, we must use unclicked items (otherwise, it would be like training a classifier with only positive labels\footnote{Recommendation from implicit data is also known as one-class collaborative filtering \citep{pan2008one}.}), i.e., $\cO$ contains all the entires in the click matrix. 

Mirroring methods for explicit data, many methods treat unclicked items as those a user does not like.  But this assumption is mistaken, and overestimates the effect of the unclicked items.  Some of these
items---many of them, in large-scale settings---are unclicked because
the user didn't see them, rather than because she chose not to
click them.  This is the crux of the problem of analyzing implicit
data: we know users click on items they like, but we do not know why
an item is unclicked.

\gls{WMF}, the standard factorization model for implicit
data, selectively downweights evidence from the click
matrix~\citep{hu2008collaborative}.  \gls{WMF} uses a simple heuristic where all
unobserved user-item interactions are equally downweighted vis-a-vis the
observed interactions. Under \gls{WMF} an observation is generated from:
\begin{align*} 
y_{ui} &\sim \mathcal{N}(\boldsymbol\theta_u^\top\boldsymbol\beta_i, c^{-1}_{y_{ui}}),
\end{align*}
where the ``confidence'' $c$ is set such that $c_1 > c_0$. This dependency between a
click and itself is unorthodox; because of it \gls{WMF} is not a generative
model. As we will describe in \Cref{sec:model} we obtain a proper generative
model by adding an exposure latent variable. 

\gls{WMF} treats the collaborative filtering problem with implicit data as a
regression problem. Concretely, consumed user-item pairs are assigned a
value of one and unobserved user-item pairs are assigned a value of zero.
Bayesian personalized ranking (BPR) \citep{rendle2009bpr,
rendle2014improving} instead treats the problem as a one of ranking
consumed user-item pairs above unobserved pairs.
In a similar vein, the weighted approximate-ranking pairwise (WARP) loss
proposed in \citet{weston2011wsabie} approximately optimizes Precision@$k$. 
To deal with the non-differentiable nature of the ranking
loss, these methods typically design specific (stochastic optimization)
methods for parameter estimation.

%with closed-form
%model inference, these methods typically require specifically designed
%stochastic method for parameter estimation.   

\section{Causal inference}\label{chpt:background:sec:causal}

\PP \citep{morgan2014counterfactuals}

\PP \citep{imbens2015causal}

\PP Potential outcomes \citep{rubin1974ece}

\PP Graphical model \citep{pearl2009causality}



